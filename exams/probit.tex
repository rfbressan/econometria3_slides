% Options for packages loaded elsewhere
\PassOptionsToPackage{unicode}{hyperref}
\PassOptionsToPackage{hyphens}{url}
%
\documentclass[
]{article}
\usepackage{amsmath,amssymb}
\usepackage{lmodern}
\usepackage{iftex}
\ifPDFTeX
  \usepackage[T1]{fontenc}
  \usepackage[utf8]{inputenc}
  \usepackage{textcomp} % provide euro and other symbols
\else % if luatex or xetex
  \usepackage{unicode-math}
  \defaultfontfeatures{Scale=MatchLowercase}
  \defaultfontfeatures[\rmfamily]{Ligatures=TeX,Scale=1}
\fi
% Use upquote if available, for straight quotes in verbatim environments
\IfFileExists{upquote.sty}{\usepackage{upquote}}{}
\IfFileExists{microtype.sty}{% use microtype if available
  \usepackage[]{microtype}
  \UseMicrotypeSet[protrusion]{basicmath} % disable protrusion for tt fonts
}{}
\makeatletter
\@ifundefined{KOMAClassName}{% if non-KOMA class
  \IfFileExists{parskip.sty}{%
    \usepackage{parskip}
  }{% else
    \setlength{\parindent}{0pt}
    \setlength{\parskip}{6pt plus 2pt minus 1pt}}
}{% if KOMA class
  \KOMAoptions{parskip=half}}
\makeatother
\usepackage{xcolor}
\usepackage[top=4cm, headheight=2cm]{geometry}
\usepackage{graphicx}
\makeatletter
\def\maxwidth{\ifdim\Gin@nat@width>\linewidth\linewidth\else\Gin@nat@width\fi}
\def\maxheight{\ifdim\Gin@nat@height>\textheight\textheight\else\Gin@nat@height\fi}
\makeatother
% Scale images if necessary, so that they will not overflow the page
% margins by default, and it is still possible to overwrite the defaults
% using explicit options in \includegraphics[width, height, ...]{}
\setkeys{Gin}{width=\maxwidth,height=\maxheight,keepaspectratio}
% Set default figure placement to htbp
\makeatletter
\def\fps@figure{htbp}
\makeatother
\setlength{\emergencystretch}{3em} % prevent overfull lines
\providecommand{\tightlist}{%
  \setlength{\itemsep}{0pt}\setlength{\parskip}{0pt}}
\setcounter{secnumdepth}{-\maxdimen} % remove section numbering
% Preamble to UDESC/Esag exams

\usepackage{fancyhdr}

%..This section controls the header-footer layout of the document
\pagestyle{fancy}
\fancyhf{}
\fancyhead[L]{\includegraphics[height=2cm]{UdescEsag.jpeg}}
\fancyhead[R]{\textit{Prova de Econometria 3\\xx de setembro de 2022}}

% \fancyfoot[L]{}

\fancypagestyle{plain}{\pagestyle{fancy}}
% End of exam_preamble --------------------------------
\usepackage{booktabs}
\usepackage{longtable}
\usepackage{array}
\usepackage{multirow}
\usepackage{wrapfig}
\usepackage{float}
\usepackage{colortbl}
\usepackage{pdflscape}
\usepackage{tabu}
\usepackage{threeparttable}
\usepackage{threeparttablex}
\usepackage[normalem]{ulem}
\usepackage{makecell}
\usepackage{xcolor}
\usepackage{siunitx}
\newcolumntype{d}{S[input-symbols = ()]}
\ifLuaTeX
  \usepackage{selnolig}  % disable illegal ligatures
\fi
\IfFileExists{bookmark.sty}{\usepackage{bookmark}}{\usepackage{hyperref}}
\IfFileExists{xurl.sty}{\usepackage{xurl}}{} % add URL line breaks if available
\urlstyle{same} % disable monospaced font for URLs
\hypersetup{
  pdftitle={Econometria III},
  pdfauthor={Rafael Bressan},
  hidelinks,
  pdfcreator={LaTeX via pandoc}}

\title{Econometria III}
\usepackage{etoolbox}
\makeatletter
\providecommand{\subtitle}[1]{% add subtitle to \maketitle
  \apptocmd{\@title}{\par {\large #1 \par}}{}{}
}
\makeatother
\subtitle{Modelos de Escolha Qualitativa\\
Questões de Prova}
\author{Rafael Bressan}
\date{2022-09-08}

\begin{document}
\maketitle

\textbf{Orientações:}

\begin{itemize}
\tightlist
\item
  A prova tem duração de 1 hora e 30 minutos
\item
  As respostas devem ser escritas a caneta (azul ou preta)
\item
  A interpretação dos enunciados faz parte da prova
\item
  Os cálculos devem estar completos, legíveis e serem apresentados junto
  com as respostas
\item
  Colou, zerou
\item
  DICA: Respire fundo e não entre em pânico
\end{itemize}

\hypertarget{questuxe3o-1}{%
\subsection{Questão 1}\label{questuxe3o-1}}

Em um modelo de regressão logística estamos interessados em estimar a
probabilidade de resposta positiva, ou seja:

\[E[Y|\symbf{x}]=p(\symbf{x})=\frac{\exp(\symbf{x\beta})}{1+\exp(\symbf{x\beta})}\]
onde \(\symbf{x\beta}\) é a notação vetorial para uma combinação linear
dos elementos de um vetor \(\symbf{x}\) que contém a unidade como seu
primeiro elemento.

\begin{enumerate}
\def\labelenumi{\alph{enumi})}
\tightlist
\item
  Mostre que a especificação não-linear do logit para a probabilidade de
  resposta positiva implica em um modelo \textbf{linear} para
  \emph{log-odds ratio} (razão entre probabilidades de sucesso e
  fracasso em logarítimo),
  \(\log\left(\frac{p(\symbf{x})}{1-p(\symbf{x})}\right)\).
\end{enumerate}

\hypertarget{questuxe3o-1-1}{%
\subsection{Questão 1}\label{questuxe3o-1-1}}

Suponha que coletamos dados para um grupo de alunos em uma aula de
econometria 3 com variáveis \(x_1\) = horas estudadas, \(x_2\) = nota
final em econometria 1 e \(y\) = nota acima de 9,0 (conceito A).
Ajustamos uma regressão logística e os coeficientes estimados são,
\(\beta_0\) = −6, \(\beta_1\) = 0,05 e \(\beta_2\) = 0,5.

\begin{enumerate}
\def\labelenumi{\alph{enumi})}
\item
  Estime a probabilidade de um estudante que estuda 40 horas e teve 7,0
  como nota de econometria 1 de receber um A.
\item
  Quantas horas o aluno da parte a) precisaria estudar para ter 50\% de
  chance de tirar A na disciplina?
\end{enumerate}

\hypertarget{questuxe3o-1-2}{%
\subsection{Questão 1}\label{questuxe3o-1-2}}

Seja \(\mathbf{z_1}\) um vetor de variáveis, \(z_2\) uma variável
contínua e seja \(y\) e \(d_1\) variáveis binárias.

No modelo
\(P(y=1|\mathbf{z_1}, z_2)=\Phi(\mathbf{z_1\delta_1}+\gamma_1z_2+\gamma_2z_2^2)\)

\begin{enumerate}
\def\labelenumi{\alph{enumi})}
\item
  Como você estimaria este modelo? Por que?
\item
  Encontre o efeito parcial de \(z_2\) na probabilidade de resposta.
\end{enumerate}

No modelo
\(P(y=1|\mathbf{z_1}, z_2, d_1)=\Phi(\mathbf{z_1\delta_1}+\gamma_1z_2+\gamma_2d_1+\gamma_3z_2d_1)\)

\begin{enumerate}
\def\labelenumi{\alph{enumi})}
\setcounter{enumi}{2}
\tightlist
\item
  Como você mediria o efeito de \(d_1\) na probabilidade de resposta?
\end{enumerate}

Suponha que tenhamos uma amostra de tamanho \(N\). Após estimar os
parâmetros pelo método mais adequado:

\begin{enumerate}
\def\labelenumi{\alph{enumi})}
\setcounter{enumi}{3}
\tightlist
\item
  Como você calcularia o efeito parcial médio (APE) de \(z_2\) no modelo
  \(P(y=1|\mathbf{z_1}, z_2)=\Phi(\mathbf{z_1\delta_1}+\gamma_1z_2+\gamma_2z_2^2)\)?
  Escreva a equação e explique-a.
\end{enumerate}

\hypertarget{questuxe3o-2}{%
\subsection{Questão 2}\label{questuxe3o-2}}

Um modelo de variável dependente binária pode ser racionalizado através
de um modelo de variável latente. Seja
\(y_i^*=\beta_0 + \beta_1 x_i+e_i\) uma variável latente (não observada)
e \(y_i=\mathbb{1}\{y_i^* > 0\}\) a variável observada. Considere que a
distribuição do erro \(e\) seja \(G\), simétrica ao redor de zero.

\begin{enumerate}
\def\labelenumi{\alph{enumi})}
\tightlist
\item
  Mostre que \(P(y=1|x)=G(\beta_0 + \beta_1 x_i)\). Ou seja, a variável
  observada \(y_i\) segue um modelo binomial. DICA: use o fato que para
  distribuições simétricas ao redor de zero \(1-G(-z)=G(z)\),
  \(z\in\mathbb{R}\).
\end{enumerate}

Suponha agora que \(e|x,c \sim N(0,1)\) e exista uma variável
explicativa não-observável que é \textbf{independente} de \(x\). O
modelo estrutural correto seria este

\[P(y_i=1|x_i, c_i)=\Phi(\beta_0 + \beta_1 x_i + \gamma c_i)\]

este problema é conhecido como heterogeneidade negligenciada. Considere
que \(c\sim N(0,\tau^2)\) e independente de ambos \(x\) e \(e\).

\begin{enumerate}
\def\labelenumi{\alph{enumi})}
\setcounter{enumi}{1}
\item
  Escreva o problema na forma de variável latente. Uma expressão para
  \(y^*\) e outra para \(y\).
\item
  Qual a distribuição do erro composto \(\gamma c + e\)? Tipo, média e
  variância.
\item
  Mostre que neste caso, o que estimamos é
  \(P(y_i=1|x_i)=\Phi((\beta_0 + \beta_1 x_i)/\sigma)\), onde
  \(\sigma^2\) é a variância do erro composto.
\item
  Interprete o resultado do item anterior quanto a viés de estimação de
  \(\beta_1\).
\end{enumerate}

\hypertarget{questuxe3o-3}{%
\subsection{Questão 3}\label{questuxe3o-3}}

Temos acesso aos dados de inadimplência de cartão de crédito de 5.000
pessoas e gostaríamos de modelar a probabilidade de inadimplência como
função do saldo em aberto no cartão. Para tanto, recorremos ao modelo de
probabilidades lineares (MPL) e ao modelo Logit. Os resultados da
regressão são apresentados na Tabela abaixo:

\begin{table}
\centering
\begin{tabular}[t]{lcc}
\toprule
  & MPL & Logit\\
\midrule
(Intercept) & \num{-0.072} & \num{-10.475}\\
 & (\num{0.005}) & (\num{0.502})\\
balance1K & \num{0.126} & \num{5.397}\\
 & (\num{0.005}) & (\num{0.307})\\
\midrule
Num.Obs. & \num{5000} & \num{5000}\\
\bottomrule
\multicolumn{3}{l}{\rule{0pt}{1em}Erro-padrão entre parênteses.}\\
\end{tabular}
\end{table}

onde a variável dependente é a situção devedora (1 = está inadimplente)
e \texttt{balance1K} é o saldo do cartão em milhares de Reais.

\begin{enumerate}
\def\labelenumi{\alph{enumi})}
\item
  O saldo do cartão parece ser relevante para determinar a situação de
  inadimplência? Qual a direção da previsão?
\item
  no modelo MPL, a partir de qual valor de saldo a previsão de
  probabilidade passa a ser negativa?
\item
  Sabendo que a média dos saldos e o percentil 75\% são respectivamente
  de, 0.8313723 e 1.1635191, calcule o efeito do aumento de R\$ 1.000 na
  probabilidade de inadimplência, tanto para MPL quanto para Logit,
  nestes pontos da distribuição. DICA: a distribuição logística é
  \(\Lambda(z)=\frac{\exp(z)}{1+\exp(z)}\)
\item
  Com base na matriz de confusão apresentada abaixo calcule o percentual
  cometido de erros do tipo I e II e responda, qual modelo você prefere
  utilizar e por que?
\end{enumerate}

\begin{table}[H]
\centering
\begin{tabular}[t]{lrrrr}
\toprule
\multicolumn{1}{c}{ } & \multicolumn{2}{c}{MPL} & \multicolumn{2}{c}{Logit} \\
\cmidrule(l{3pt}r{3pt}){2-3} \cmidrule(l{3pt}r{3pt}){4-5}
  & FALSE & TRUE & FALSE & TRUE\\
\midrule
FALSE & 4836 & 0 & 4819 & 17\\
TRUE & 164 & 0 & 118 & 46\\
\bottomrule
\end{tabular}
\end{table}

\hypertarget{questuxe3o-4}{%
\subsection{Questão 4}\label{questuxe3o-4}}

Quando nossa variável dependente é oriunda de um processo de contagem,
ou seja, é inteira não negativa (\(y_i \in \mathbb{Z_+}\)), costuma-se
utilizar a regressão de Poisson. A distribuição de Poisson é assim
definida:

\[P(Y=k)=\frac{e^{-\lambda}\lambda^k}{k!}, \qquad k\in\mathbb{Z_+}\]
onde o parâmetro \(\lambda\) é o valor esperado de \(Y\),
\(E[Y]=\lambda\). Na prática, sempre supomos que esta média é
condicional a variáveis explanatórias e considerando que a variável
dependente nunca assume valores negativos, uma parametrização para a sua
média condicional é:

\[E[y|x]=\exp(\beta x).\]

Suponha que temos uma amostra aleatória de \(N\) observações
independentes, \(\{(y_i, x_i)\}_{i=1}^N\), onde \(y_i\) é um processo de
contagem.

\begin{enumerate}
\def\labelenumi{\alph{enumi})}
\item
  Com base nas informações acima, monte a função de verossimilhança de
  uma regressão de Poisson.
\item
  Derive a função \emph{score} da maximização da log-verossimilhança. É
  possível resolver analiticamente esta equação?
\end{enumerate}

Agora ajustamos um modelo de regressão de Poisson ao conjunto de dados
\texttt{Bikeshare}. As variáveis explicativas para o número de ciclistas
são:

\begin{enumerate}
\def\labelenumi{\roman{enumi})}
\tightlist
\item
  \texttt{workingday}, variável dummy para dia de trabalho;
\item
  \texttt{temp}, temperatura \emph{normalizada}. Normalização é
  \((t-t_{min})/(t_{max}-t_{min})\), com \(t_{min}=-8^oC\) e
  \(t_{max}=39^oC\);
\item
  \texttt{weathersit}, variável categórica da condição climática. As
  categorias de clima são \texttt{clear}, \texttt{cloudy/misty},
  \texttt{light\ rain/snow} e \texttt{heavy\ rain/snow}.
\end{enumerate}

Os resultados são mostrados na Tabela abaixo.

\begin{table}
\centering
\begin{tabular}[t]{lc}
\toprule
  & Model 1\\
\midrule
(Intercept) & \num{3.885}\\
 & (\num{0.003})\\
workingday & \num{-0.009}\\
 & \vphantom{1} (\num{0.002})\\
temp & \num{2.129}\\
 & (\num{0.005})\\
weathersitcloudy/misty & \num{-0.042}\\
 & (\num{0.002})\\
weathersitlight rain/snow & \num{-0.432}\\
 & (\num{0.004})\\
weathersitheavy rain/snow & \num{-0.761}\\
 & (\num{0.167})\\
\midrule
Num.Obs. & \num{8645}\\
Log.Lik. & \num{-434895.777}\\
\bottomrule
\multicolumn{2}{l}{\rule{0pt}{1em}Erro-padrão entre parênteses.}\\
\end{tabular}
\end{table}

\begin{enumerate}
\def\labelenumi{\alph{enumi})}
\setcounter{enumi}{2}
\item
  Qual o efeito parcial de uma variação marginal na temperatura no
  número de ciclistas (\(PE_t\))? Encontre uma expressão analítica e
  responda se este efeito é constante. DICA:
  \(PE_t=\partial E[y|\symbf{x}]/\partial x_t\)
\item
  Qual o valor de \(PE_t\) se for um dia claro, de trabalho e com
  temperatura normalizada de 0,5?
\item
  Qual o valor do efeito de uma mudança climática, de dia claro para
  chuva forte, em um dia de descanso com a menor temperatura registrada
  nos dados?
\end{enumerate}

\hypertarget{questuxe3o-5}{%
\subsection{Questão 5}\label{questuxe3o-5}}

Demonstre que, para um modelo de regressão linear simples,

\[y_i=\beta_0+\beta_1 x_i+u_i\]

considerando que \(u_i\sim N(0, \sigma^2)\), os estimadores de Máxima
Verossimilhança são iguais aos estimadores de MQO. DICA: a função
densidade da normal é

\[
{\displaystyle f(z)={\frac {1}{\sigma {\sqrt {2\pi }}}}e^{-{\frac {1}{2}}\left({\frac {z-\mu }{\sigma }}\right)^{2}}}
\]

\end{document}
